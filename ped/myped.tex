\documentclass[12pt]{article}
\usepackage[arabic]{babel}
\usepackage[utf8]{inputenc}
\usepackage{amssymb}
\usepackage{geometry}
\usepackage{graphicx}
\usepackage{wrapfig}
\usepackage{amsmath}
\usepackage{lipsum}
\usepackage{caption}
\usepackage{ulem}
\usepackage{xcolor}
\usepackage{polyglossia}
\usepackage{tikz}
\usetikzlibrary{arrows.meta, positioning, shapes}
\setdefaultlanguage{arabic}
\setotherlanguage{english}

% Font and layout settings
\newfontfamily\arabicfont[Script=Arabic,Scale=1.2]{Amiri}
\geometry{top=2cm, bottom=2cm, left=2cm, right=2cm}

% Colors
\definecolor{titleColor}{HTML}{800000} % Maroon for titles
\definecolor{sectionColor}{HTML}{4682B4} % SteelBlue for sections
\definecolor{textHighlight}{HTML}{DAA520} % Goldenrod for highlights
\definecolor{backgroundColor}{HTML}{F0F8FF} % AliceBlue background
\definecolor{emphasisColor}{HTML}{228B22} % ForestGreen for emphasis


\begin{document}

بابكر/مالك/عثمان/على/الشيخ/محمود
 / الفكي قيلى/مالك/  
مدني /عبدالرحمن/ابوالقاسم/حميدة/حمد/حميدة/سراج/ مصباح/سليم/ رباط/غلام اللة/عائد

مقبول/احمد الزيلعي/محمود/هاشم/ مختار/علي العسكري/محمد/ابوالقاسم/محمد الزامل/اسماعيل/ احمد التقي/محمد النقي/علي الرضا/موسي الكاظم/جعفر الصادق/ محمد الباقر/ زين العابدين علي/الحسين/ بن علي كرم الله وجهه والسيدة فاطمة الزهراء بنت رسول الله صلى الله عليه وسلم 


 
هو الفكي قيلي صاحب القبة الموجوده بالعفاض الآن.. ودرس الفقه وحفظ القران علي يد جده مدني . وفي فترة الحكم التركي طلب الأتراك من جده الفكي قيلي الحضور إلى تركيا ليكون ضمن علماء البلاط التركي وديوان الافتاء الا انه فضل البقاء بالسودان..متجولا في شمال السودان يدرس علوم الدين في مناطق شمال السودان


\end{document}


