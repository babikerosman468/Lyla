\documentclass{article}   
\usepackage{graphicx}       
\usepackage{wrapfig}
\usepackage{amsmath}     
\usepackage{lipsum}       
\usepackage{caption}
\usepackage{tikz}
\usetikzlibrary{arrows.meta, positioning}
\usetikzlibrary{shapes, positioning} 
\usepackage{ulem} 
\usepackage{color} 

\title{Inguinal Hernia}
\author{B M Osman  \footnote {Al Nahada univerdity Cairo \email babikerosman@yahoo.com }}
\date{\today}
\begin{document}
\maketitle
\tableofcontents  

\section{Introduction}
An \textbf{inguinal hernia} occurs when tissue, such as part of the intestine or abdominal fat, protrudes through a weak spot in the abdominal muscles near the \textbf{inguinal canal}. It is the most common type of hernia and occurs more frequently in men than women.

\section{Types of Inguinal Hernias}
\begin{itemize}
    \item \textbf{Indirect Inguinal Hernia} – More common, often congenital, and occurs when abdominal contents protrude through the deep inguinal ring into the inguinal canal.
    \item \textbf{Direct Inguinal Hernia} – Develops due to weakness in the abdominal wall, typically in older adults, and protrudes directly through the \textit{Hesselbach's triangle}.
\end{itemize}

\section{Causes and Risk Factors}
\begin{itemize}
    \item Congenital weakness in the abdominal wall
    \item Chronic coughing or straining
    \item Heavy lifting
    \item Obesity
    \item Aging
    \item Pregnancy
    \item Previous abdominal surgery
\end{itemize}

\section{Symptoms}
\begin{itemize}
    \item A bulge in the groin or scrotum (in men)
    \item Pain or discomfort, especially when bending, lifting, or coughing
    \item A heavy or dragging sensation in the groin
    \item Swelling that increases with activity and decreases when lying down
\end{itemize}

\section{Complications}
\begin{itemize}
    \item \textbf{Incarceration} – The hernia becomes trapped and cannot be pushed back.
    \item \textbf{Strangulation} – Blood supply to the herniated tissue is cut off, leading to severe pain, nausea, vomiting, and requiring emergency surgery.
\end{itemize}

\section{Diagnosis}
\begin{itemize}
    \item Physical examination
    \item Ultrasound or CT scan (if needed for confirmation)
\end{itemize}

\section{Treatment}
\subsection{Watchful Waiting}
Small, asymptomatic hernias may not require immediate surgery.

\subsection{Surgical Repair}
Surgical intervention is recommended for symptomatic or large hernias:
\begin{itemize}
    \item \textbf{Open Hernia Repair (Herniorrhaphy/Hernioplasty)} – The surgeon makes an incision and reinforces the weak area with stitches or mesh.
    \item \textbf{Laparoscopic Hernia Repair} – A minimally invasive approach using small incisions and a camera-guided technique to place a mesh.
\end{itemize}

\end{document}

