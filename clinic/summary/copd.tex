\textcolor{blue}{Chronic Obstructive Pulmonary Disease (COPD)} is a \textcolor{red}{progressive inflammatory lung condition} primarily caused by long-term exposure to \textcolor{green}{harmful substances} like cigarette smoke. It includes two main conditions: \textcolor{purple}{chronic bronchitis} (characterized by \textcolor{orange}{inflammation and excessive mucus}) and \textcolor{purple}{emphysema} (where the \textcolor{orange}{alveoli are destroyed}). Symptoms such as \textcolor{orange}{persistent cough, shortness of breath, and wheezing} are common, and risk factors include \textcolor{green}{smoking}, \textcolor{green}{secondhand smoke}, \textcolor{green}{pollution}, and \textcolor{green}{genetic predispositions}. Diagnosis is made using \textcolor{blue}{spirometry}, \textcolor{blue}{imaging tests}, and \textcolor{blue}{blood tests}. Management includes \textcolor{blue}{lifestyle changes} like quitting smoking, \textcolor{blue}{medications} such as \textcolor{red}{bronchodilators} and \textcolor{red}{corticosteroids}, and \textcolor{blue}{surgical options} like \textcolor{red}{lung volume reduction surgery}. Vaccines for \textcolor{blue}{flu and pneumonia} are also recommended. Although COPD is \textcolor{red}{incurable}, \textcolor{blue}{early diagnosis} and treatment can \textcolor{green}{improve quality of life} and slow its progression.



