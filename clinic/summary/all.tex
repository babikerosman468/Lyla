\documentclass{article}   
\usepackage{graphicx}       
\usepackage{wrapfig}
\usepackage{amsmath}     
\usepackage{lipsum}       
\usepackage{caption}
\usepackage{tikz}
\usetikzlibrary{arrows.meta, positioning}
\usetikzlibrary{shapes, positioning} 
\usepackage{ulem} 
\usepackage{xcolor} 

\title{Clinical Practice Summary}

\author{B M Osman  \footnote {Al Nahada univerdity Cairo \email babikerosman@yahoo.com }}
\date{\today}
\begin{document}
\maketitle

\section*{Summary}

\textcolor{teal}{Chronic Venous Ulcers (CVUs)} are a type of chronic skin ulcer caused by \textcolor{blue}{venous insufficiency}, lasting more than 6 weeks, and commonly located near the \textcolor{purple}{medial malleolus}. The pathophysiology involves \textcolor{red}{venous hypertension} due to valve incompetence or obstruction, leading to \textcolor{green}{venous stasis} and fibrin cuff formation, which causes \textcolor{orange}{inflammation} and \textcolor{yellow}{tissue hypoxia}, resulting in ulceration. Risk factors include \textcolor{green}{chronic venous insufficiency}, \textcolor{green}{deep vein thrombosis}, \textcolor{green}{varicose veins}, \textcolor{green}{obesity}, and \textcolor{green}{prolonged standing}. The ulcers present as \textcolor{orange}{irregular, shallow wounds} with yellow slough and granulation, accompanied by \textcolor{brown}{hemosiderin deposition} and \textcolor{red}{lipodermatosclerosis}. Pain is typically \textcolor{orange}{mild to moderate} and worsens with standing but improves with leg elevation. Diagnosis includes \textcolor{blue}{clinical examination}, the \textcolor{purple}{Ankle-Brachial Index}, and \textcolor{blue}{duplex ultrasound}. Management involves \textcolor{blue}{conservative treatments} like \textcolor{purple}{compression therapy}, \textcolor{blue}{wound care}, and \textcolor{blue}{skin care}, while \textcolor{red}{medical treatments} include \textcolor{orange}{pentoxifylline} and \textcolor{orange}{antibiotics} for infections. \textcolor{green}{Surgical options} like endovenous ablation or \textcolor{purple}{skin grafting} may be considered for advanced cases. The \textcolor{blue}{prognosis} suggests that healing may take 3–6 months, with a 70\% recurrence rate in 5 years without proper compression therapy. Preventive measures include \textcolor{blue}{compression stockings} and \textcolor{green}{regular exercise}.   

\\
\bigskip

\textcolor{teal}{Chronic Obstructive Pulmonary Disease (COPD)} is a \textcolor{red}{progressive inflammatory lung condition} primarily caused by long-term exposure to \textcolor{green}{harmful substances} like cigarette smoke. It includes two main conditions: \textcolor{purple}{chronic bronchitis} (characterized by \textcolor{orange}{inflammation and excessive mucus}) and \textcolor{purple}{emphysema} (where the \textcolor{orange}{alveoli are destroyed}). Symptoms such as \textcolor{orange}{persistent cough, shortness of breath, and wheezing} are common, and risk factors include \textcolor{green}{smoking}, \textcolor{green}{secondhand smoke}, \textcolor{green}{pollution}, and \textcolor{green}{genetic predispositions}. Diagnosis is made using \textcolor{blue}{spirometry}, \textcolor{blue}{imaging tests}, and \textcolor{blue}{blood tests}. Management includes \textcolor{blue}{lifestyle changes} like quitting smoking, \textcolor{blue}{medications} such as \textcolor{red}{bronchodilators} and \textcolor{red}{corticosteroids}, and \textcolor{blue}{surgical options} like \textcolor{red}{lung volume reduction surgery}. Vaccines for \textcolor{blue}{flu and pneumonia} are also recommended. Although COPD is \textcolor{red}{incurable}, \textcolor{blue}{early diagnosis} and treatment can \textcolor{green}{improve quality of life} and slow its progression.
\\
\bigskip


\textcolor{teal}{Inguinal hernia} involves the protrusion of abdominal contents near the \textcolor{purple}{inguinal canal}, more common in \textcolor{teal}{males}. It may be \textcolor{orange}{indirect} (congenital, via the deep ring) or \textcolor{red}{direct} (acquired, through Hesselbach’s triangle). \textcolor{brown}{Risk factors} include congenital weakness, \textcolor{brown}{heavy lifting}, \textcolor{brown}{chronic cough}, and \textcolor{brown}{obesity}. Symptoms range from a \textcolor{violet}{groin bulge} and \textcolor{violet}{discomfort} to activity-related swelling. Major concerns are \textcolor{red}{incarceration} (non-reducible hernia) and \textcolor{red}{strangulation} (compromised blood supply), requiring urgent surgery. Diagnosis is largely \textcolor{blue}{clinical}, with imaging if needed. Management options include \textcolor{green}{observation} for mild cases and \textcolor{green}{surgical repair}—either \textcolor{green}{open} or \textcolor{green}{laparoscopic}—especially for symptomatic or complicated hernias.

\\
\bigskip


\textcolor{teal}{Lipomas} are common \textcolor{green}{benign tumors} composed of fatty tissue, often found on the \textcolor{blue}{upper arms}, \textcolor{blue}{thighs}, or \textcolor{blue}{torso}. They are typically \textcolor{purple}{soft}, \textcolor{orange}{slow-growing}, and \textcolor{gray}{painless}. Though the exact cause is unclear, contributing factors include \textcolor{brown}{genetics}, \textcolor{brown}{age}, \textcolor{brown}{trauma}, and syndromes like \textcolor{red}{Gardner’s Syndrome}. Diagnosis is primarily \textcolor{cyan}{clinical}, with imaging or biopsy used to rule out malignancy. Treatment may not be needed unless the lipoma causes \textcolor{magenta}{discomfort}, grows, or poses \textcolor{magenta}{cosmetic concerns}. Options include \textcolor{olive}{surgical excision}, \textcolor{olive}{liposuction}, or \textcolor{olive}{steroid injections}. Seek medical care if the lipoma becomes \textcolor{red}{painful}, enlarges quickly, or shows signs of \textcolor{red}{infection}.

\\
\bigskip

\textcolor{teal}{Bicuspid aortic valve stenosis} is a congenital heart condition where the aortic valve has \textcolor{blue}{two cusps instead of three}, leading to early and progressive \textcolor{orange}{valve narrowing}. It often presents in \textcolor{purple}{younger adults} with symptoms like \textcolor{red}{chest pain}, \textcolor{red}{dyspnea}, \textcolor{red}{syncope}, or may be asymptomatic until severe. The abnormal valve structure accelerates \textcolor{brown}{calcification} and worsens with age. Diagnosis is confirmed with \textcolor{cyan}{echocardiography}, assessing valve anatomy and stenosis severity. Management includes \textcolor{olive}{regular monitoring}, and in advanced cases, \textcolor{green}{surgical valve replacement} or \textcolor{green}{transcatheter aortic valve implantation (TAVI)} is indicated.




\end{document}

