
\section*{Summary}
Lipomas are common benign tumors made of fatty tissue, typically appearing on the upper arms, thighs, or torso. They are soft, slow-growing, and usually painless. While the exact cause is unknown, factors like genetics, age, trauma, and certain genetic syndromes (e.g., Gardner’s Syndrome) may contribute. Diagnosis is often clinical but may involve imaging or biopsy to exclude malignancy. Treatment is not always necessary unless the lipoma causes discomfort, grows, or raises cosmetic concerns; options include surgical excision, liposuction, or steroid injections. Medical attention is advised if the lipoma changes rapidly, becomes painful, or shows signs of infection.



