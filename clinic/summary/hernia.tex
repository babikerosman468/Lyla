\documentclass{article}
\usepackage{amsmath}
\usepackage{enumitem}

\title{Summary: Inguinal Hernia}
\author{B M Osman}
\date{\today}

\begin{document}
\maketitle

\section*{Definition}
Inguinal hernia is a condition where abdominal contents protrude through a weak area near the inguinal canal, predominantly affecting males.

\section*{Types}
\begin{itemize}[noitemsep]
  \item \textbf{Indirect}: Congenital, through deep inguinal ring.
  \item \textbf{Direct}: Acquired, through Hesselbach’s triangle.
\end{itemize}

\section*{Risk Factors}
Congenital wall weakness, heavy lifting, chronic cough, obesity, aging, pregnancy, and previous surgeries.

\section*{Symptoms}
Groin bulge, pain during movement, dragging sensation, activity-related swelling.

\section*{Complications}
\begin{itemize}[noitemsep]
  \item \textbf{Incarceration}: Hernia trapped, non-reducible.
  \item \textbf{Strangulation}: Blood supply cutoff—surgical emergency.
\end{itemize}

\section*{Diagnosis}
Primarily clinical; imaging (ultrasound/CT) if unclear.

\section*{Management}
\begin{itemize}[noitemsep]
  \item \textbf{Observation}: For small, asymptomatic hernias.
  \item \textbf{Surgical Repair}:
    \begin{itemize}
        \item \textbf{Open Repair}: Mesh/stitches.
        \item \textbf{Laparoscopic Repair}: Minimally invasive mesh placement.
    \end{itemize}
\end{itemize}

\end{document}



