\documentclass[12pt]{article}
\usepackage{amssymb}
\usepackage{geometry}
\usepackage{graphicx}
\usepackage{hyperref}
\usepackage{wrapfig}
\usepackage{amsmath}
\usepackage{lipsum}
\usepackage{caption}
\usepackage{ulem}
\usepackage{xcolor}
\usepackage{tikz}
\usetikzlibrary{arrows.meta, positioning, shapes}

\usepackage{sectsty}
\sectionfont{\color{sectionColor}}       % Color for \section
\subsectionfont{\color{subsectionColor}}


\definecolor{titleColor}{HTML}{800000}        % Maroon
\definecolor{sectionColor}{HTML}{4682B4}      % SteelBlue
\definecolor{textHighlight}{HTML}{DAA520}     % Goldenrod
\definecolor{backgroundColor}{HTML}{F0F8FF}   % AliceBlue
\definecolor{emphasisColor}{HTML}{228B22}     % ForestGreen
\definecolor{subsectionColor}{HTML}{6A5ACD}   % SlateBlue


\title{\textcolor{titleColor}{\textbf{\Huge \textbf{ Reflections on 45 Years of Working with SAS }}}}

\author{B. M. Osman\thanks{Al Nahada University, Cairo \\ \texttt{babikerosman@yahoo.com}}}
\date{\today}
\begin{document}
\maketitle


\section*{Introduction}
For more than four decades, my academic and research journey has been closely tied to the use of \textbf{SAS Institute products}. Over these 45 years, I have witnessed the evolution of computing environments---from mainframes to personal computers, from the emergence of open-source ecosystems to the era of artificial intelligence. Throughout this period, SAS has remained a consistent and reliable companion for research, teaching, and publication.

\section{The Value of SAS in Research}
SAS offers one of the most stable, rigorous, and reproducible environments for data analysis. As an academic, I have relied on its wide array of statistical procedures, data handling capabilities, and reproducibility standards, which make it uniquely suited for high-stakes research and publications. 

The integration of SAS with tools such as \textbf{Vim} and \textbf{\LaTeX} has provided me with a complete scholarly workflow:  
\begin{itemize}
  \item \textbf{SAS} for data management, modeling, and analytics.  
  \item \textbf{Vim} for fast, distraction-free editing of scripts and notes.  
  \item \textbf{\LaTeX} for producing professional publications and documentation.  
\end{itemize}

Together, this triad forms what I call a \textit{research habitat}, optimized for productivity, creativity, and precision.

\section{Adaptation in the Age of AI}
The last decade has introduced dramatic changes in computational science with the rise of \textbf{artificial intelligence}, machine learning, and mobile computing. Rather than diminishing SAS, these changes have created opportunities for new integrations:  
\begin{itemize}
  \item Running SAS workflows alongside Python and AI models, bridging classical statistical rigor with modern machine learning.  
  \item Deploying SAS environments in \textbf{Termux Android emulators}, enabling mobile experimentation and analysis in the palm of one’s hand.  
  \item Enhancing efficiency through hybrid workflows where SAS is complemented by open-source libraries and cloud-based AI platforms.  
\end{itemize}

These adaptations make working with SAS today even more enjoyable, blending tradition with innovation.

\section{Personal Reflections}
My career with SAS is not simply about software---it is about continuity, adaptability, and intellectual companionship. Few tools in the scientific world can boast of maintaining relevance for nearly half a century. SAS has been both a foundation and a bridge:  
\begin{quote}
``A foundation in providing reliable, reproducible, and trusted analysis, and a bridge in connecting traditional statistics with emerging technologies such as AI and mobile emulation.''
\end{quote}

\section*{Conclusion}
After 45 years of daily use, I can affirm that SAS remains the most helpful environment for academic and research work. Combined with Vim, \LaTeX, and modern AI-driven platforms, it creates an enjoyable, efficient, and forward-looking ecosystem for scientific inquiry.  

\vfill
\begin{center}
\textit{--- End of Reflection ---}
\end{center}

\end{document}


