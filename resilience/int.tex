\section{Introduction}
Traumatic events lead to severe mental health problems (including anxiety, depression, and post-traumatic stress disorder) in approximately 10\% of individuals exposed to trauma, with an additional 10\% developing behavioral issues that affect their daily functioning, as estimated by the World Health Organization (1,2).

Resilience is an important determinant of mental health outcomes after traumatic events. It is defined as the ability to recover after hard times and gain strength from the recovery process (3). Within healthcare systems resilience is crucial to maintain services during crisis (4). Factors proved by studies to foster Resilience include: self-protection measures, internal coping mechanisms such as faith and patriotism, and external support including international assistance and training (4).

A high perceived sense of danger among healthcare workers in conflict zones was found to lower their resilience and capacity to recover, elevate their stress levels, and affect their report to duty (5-7). Studies have linked a high perceived sense of danger to higher absenteeism rates (7).

An outbreak of armed conflict erupted in Sudan on April 15, 2023, triggering a severe humanitarian crisis (8). It damaged the health system as only one-third of healthcare facilities in conflict zones have left operational after the destruction of hospitals, disruption of medical supplies, migration of healthcare professionals, and the increased cost of the already underfunded healthcare sector (9).

This conflict and its consequences left healthcare workers operating under immense psychological, physical and professional stress: poor working conditions, Lack of medical supplies, insufficient infrastructure, interrupted salaries and physical threats including abduction, targeted attacks, and insecurity (9,10). Reports of military attacks on health care workers in Sudan reflect the broader global trend of targeting healthcare workers during conflicts, compromising the health system’s resilience (6).

Conflict indeed created intense pressure on healthcare workers. Stressful situations generally impair workers’ ability to function effectively, but their individual resilience during these times significantly influences their functionality (11). Limited literature discusses the resilience and coping mechanisms of Sudanese healthcare workers during stressful events, and to the best of our knowledge, no studies have discussed their resilience and perceived sense of danger during wartime; however, the conflict continues to challenge their mental well-being and safety during violence and threats. Here comes the purpose of our research: to highlight the struggle of Sudanese healthcare workers and study the factors that might potentially improve their resilience or decrease it, and what role the perceived sense of danger plays in this issue.

