Resilience, Perceived Danger, and Work Attendance of Sudanese Healthcare Workers In War Zones: The April 2023 Armed Conflict


1-Maryam Mohamed (corresponding author)
mariam.jack55@yahoo.com
University of Khartoum, Khartoum, Sudan 

2-Sami Hassan 
sami.mo.al.ha@gmail.com
University of Khartoum, Khartoum, Sudan 

3-Abubakr Mahmoud
abubakr.m.i.mahmoud95@gmail.com
University of Khartoum , Khartoum, Sudan.

4-Laila Osman
Lylababiker@gmail.com
University of Khartoum, Khartoum, Sudan.

5-Huda Ahmed 
hudamagdi16@gmail.com
University of Khartoum, Khartoum, Sudan.

6-Alameen Osama Abdalgadir Abass
ameenosama52@gmail.com
Alnau hospital, Omdurman, Sudan 

7-Elmoez abdoalazeem Omer mhadey
elmo3z234@gmail.com
University of Khartoum, Khartoum, Sudan.



8-Elharith Hatim Elhag Abdelwahid
elharithhatim@gmail.com
University of Science and Technology, Khartoum, Sudan 
Albuluk Specialized Hospital for Children

9-Hiba Hamad 
hibakamil2@gmail.com
University of Khartoum, Khartoum, Sudan.

10-Mohamed Osman Abdalla Osman
mo929569609@gmail.com
Karary university, Omdurman, Sudan 















Abstract:
Background : Traumatic war events can  impair healthcare workers functioning and psychological wellbeing.we aim to assess Sudanese healthcare workers resilience, attendance, and perceived sense of danger in active conflict zones.
Methods : we conducted a cross-sectional study  (January and March 2025) in four major public hospitals  operating in compact zones , Omdurman city .Participants included doctors, nurses,
lab technicians, and Pharmacists , we used a  self administered questionnaire assessing: sociodemographics  , traumatic exposure,Connor-Davidson Resilience Scale-10 , Solomon and Prager adapted Sense of Danger Inventory , and work attendance . Non-parametric tests, chi-square/Fisher’s exact tests, Spearman’s correlation, and multivariable linear regression were performed using SPSS v27, with significance set at p < 0.05.
Results :Among 325 participants (mean age 28.3 ± 4.9 year ;  female (57.2%), (60%)  reported  property damage , with personal (21%) or family  (49%) injury,and workplace security escalations (41%), while utilization of support services was limited.
 Attendance  was higher among males, single workers, those with crisis training, and  without personal or family injuries . Median resilience score was 24.00 (IQR = 12.00) with (68.6%)  having low resilience,  associated with male gender, higher religiosity, crisis training, full attendance, and hospital assignment. Median sense of danger was high (Md = 14.0), particularly among pharmacists/lab technicians and  Albuluk staff, it was associated with security escalations and lack of group activities . Resilience and sense of danger showed a weak but significant positive correlation (rho = 0.178, p = 0.001).
conclusion :Sudanese healthcare workers in conflict zones exhibited low resilience and  high Perceived danger,  influenced by  gender, crisis training, and workplace security. These findings underscore the  need for urgent  mental health and safety interventions to sustain this critical workforce .


Keywords: Resilience, Sense of Danger,Work attendance, healthcare workers,armed conflict , war , Sudan 

Introduction: 
Traumatic events lead to severe mental health problems ( including anxiety, depression, and post-traumatic stress disorder ) in approximately 10% of individuals exposed to trauma, with an additional 10% developing behavioral issues that affect their daily functioning, as estimated by the World Health Organization (1,2). 
Resilience is an important determinant of mental health outcomes after traumatic events. It is defined as the ability to recover after hard times and gain strength from the recovery process(3). Within healthcare systems resilience is crucial to maintain services during crisis  (4) Factors proved by studies to foster Resilience include: self-protection measures , internal coping mechanisms such as faith and patriotism , and external support including international assistance and training (4) 
A high perceived sense of danger among healthcare workers in conflict zones was found to lower their resilience and capacity to recover, elevate their stress levels, and affect their report to duty (5-7). Studies have linked a high perceived sense of danger to higher absenteeism rates (7). 
An outbreak of armed conflict erupted in Sudan on April 15,2023, triggering a severe humanitarian crisis (8). It damaged the health system as only one-third of healthcare facilities in conflict zones have left operational after the destruction of hospitals, disruption of medical supplies, migration of healthcare professionals, and the increased cost of the already underfunded healthcare sector(9).
 This conflict and its consequences left healthcare workers operating under immense psychological,  physical and professional stress : poor working conditions , Lack of medical supplies , insufficient infrastructure,  interrupted salaries and physical threats including abduction ,targeted attacks, and insecurity (9,10). Reports of military attacks on health care workers in Sudan reflect the broader global trend of targeting healthcare workers during conflicts, compromising the health system’s resilience (6). 
Conflict indeed created intense pressure on healthcare workers. Stressful situations generally impair workers’ ability to function effectively, but their individual resilience during these times significantly influences their functionality (11). Limited literature discusses the resilience and coping mechanisms of Sudanese healthcare workers during stressful events, and to the best of our knowledge, no studies have discussed their resilience and perceived sense of danger during wartime; however, the conflict continues to challenge their mental well-being and safety during violence and threats. Here comes the purpose of our research: to highlight the struggle of Sudanese healthcare workers and study the factors that might potentially improve their resilience or decrease it, and what role the perceived sense of danger plays in this issue.

Method
 Study Design and Setting 
This cross-sectional study was conducted from January to March 2025. The research was carried out in four major public hospitals in Omdurman city, Khartoum state: Al Nau Teaching Hospital, Alsaudi Teaching Hospital, Albuluk Teaching Hospital, and BirAlwaldeen Teaching Hospital.
The unique context of this study is that these facilities were the only public hospitals that remained almost fully operational in Khartoum State three cities ( Khartoum , Bahri , and Omdurman ) during the war that erupted in April 2023. Despite their continued operation, these hospitals were located within or near the perimeter of direct military clashes. This proximity exposed them to frequent and continuous indiscriminate shelling and missile impacts in their vicinity and, in some cases, directly on their buildings. This critical security situation imposed immense operational and psychological challenges on both medical staff and patients.

Study Population 
This study included healthcare professionals including doctors, nurses, laboratory technicians and pharmacists working in the selected hospitals during the study period who had been employed during the war for at least one month. Individuals were excluded if they were non-medical volunteers, unregistered house officers,declined to participate or were unable to provide informed consent .No discrimination regarding gender, or ethnicity was made .

Sample Size and Sampling Technique 
A stratified random allocation  was employed to ensure proportional representation of healthcare professionals across the four hospitals included in the study. This approach was chosen due to the heterogeneous distribution of staff among the hospitals, thereby enhancing the generalizability and representativeness of the findings.
Stratification and Population Distribution
The total population of eligible healthcare professionals (N = 2,030) was stratified by hospital as follows:
- Alnau Hospital: 700 professionals
- Alsaudi Hospital: 440 professionals
- Albuluk Hospital: 540 professionals
- Bir Alwaldeen Hospital: 350 professionals
The required sample size was calculated using the standard formula for proportions with a 95% confidence level and 5% margin of error:
n = (Z^2 * p * (1-p)) / e^2 = (1.96^2 * 0.5 * (1-0.5)) / 0.05^2 = 384
To adjust for the finite population, the sample size was corrected using:
n_adjusted = n / (1 + (n - 1) / N) = 384 / (1 + 383 / 2030) ≈ 322
Proportional Allocation by Stratum
The adjusted sample size was distributed proportionally across the four hospitals using the formula:
n_i = (N_i * n_adjusted) / N
Resulting in the following allocations:
- Alnau Hospital: 132 participants
- Alsaudi Hospital: 86 participants
- Albuluk Hospital: 103 participants
- Bir Alwaldeen Hospital: 66 participants
On the hospital level participants were selected conveniently due to the escalating events and the sensitivity of hospital staff personnel information. 

Data Collection and Management 
Data were collected using a self-administered  questionnaire adopted from previous similar studies and validated scales( 7,12,13)  . Participants  were provided with the google form version of the questionnaire  and given the option to fill it in person or later through their WhatsApp or telegram  . All questionnaires were anonymous. The questionnaire consisted of five sections: 
Sociodemographic: included questions about general background information and professional status.
Exposure to terrorist events: 9 Questions about past experiences with terrorism, including damage to property, physical or psychological injuries, exposure at previous workplaces, and utilization of support services.(7,2)
Individual Resilience: Assessed with the 10-item Connor-Davidson Resilience Scale (CD-RISC-10)(12,2), a  5-point Likert scale (0–4) measuring aspects such as flexibility and optimism. Total scores range from 0 to 40, with higher scores reflecting greater resilience. The scale's reliability was high (α= .90).
Sense of Danger: Evaluated using the six-item Solomon and Prager Sense of Danger scale (13,2). This 5-point Likert scale (0–4) measured perceived danger across personal, family, workplace, and national domains. The reliability for this measure was high (α = .85).
Work attendance: recorded as full/ partial or totally absent using three items and additional questions about factors that made it difficult or easier to report to work.(7,3).

Statistical Analysis
 Data were analyzed using SPSS software, version 27. Continuous outcomes (resilience and sense-of-danger scores), which were non-normally distributed, are summarized as medians with interquartile ranges (IQR), while categorical variables are presented as frequencies and percentages. Group comparisons were performed using the Mann–Whitney U test for predictors with two levels, and the Kruskal–Wallis H test with Bonferroni-adjusted post hoc comparisons for predictors with three or more levels. Associations between categorical factors and work attendance were assessed using the chi-square test or Fisher’s exact test when expected cell counts were <5. The correlation between resilience and perceived danger was evaluated using Spearman’s rank-order correlation. Finally, two multivariable linear regression models (one for each outcome) were constructed, with predictors selected based on theoretical considerations. Model assumptions, including linearity, homoscedasticity, absence of multicollinearity, and independence of residuals, were checked and met. Statistical significance was set at α = 0.05.

Results
Sociodemographic Characteristics
The study included 325 participants, with a mean age of 28.3 years (SD = 4.9). The majority were female (57.2%), single (76.3%), without children (63.4%), and living with family (83.7%). Most reported a monthly income below 500,000 SDG (85.2%) and moderate perceived religiosity (77.5%). Only two participants were of non-Sudanese nationality. Detailed demographic characteristics are presented in Table 1.
Regarding professional roles, participants were distributed across various medical sectors, with house officers (27.7%) and medical officers (25.7%) being the most common. The majority (57.2%) relied on public transportation to reach work. Only 39.1% had received training in crisis management, and 34.2% were based at Alnau Hospital. Additional details on professional profiles are provided in Table 2.
Table 1: Demographics
Variable
n
%

Age



 18–40                                                
314
96.6%

 41 and above                                         
11
3.4%

Gender  



 Male                                                 
139
42.8%

 Female                                               
186
57.2%

Marital Status  



 Single       
248
76.3%

 Married   
71
21.8%

 Divorced     
3
0.9%

 Widowed 
3
0.9%

Do you have dependents in your household?  



 Yes                                                  
138
42.5%

 No                                                   
187
57.5%

Children’s age



   less than 2 years 
35    
10.8%

   2-10
69    
21.2%

   11-17 
28    
8.6%

   18+
15    
4.6%

   No children
206
63.4%

Perceived religiosity 



 Low                                                  
21
6.5%

 Moderate                                             
252
77.5%

 High                                                
52
16.0%

Nationality 



 Sudanese 
323
99.4%

 Non-Sudanese 
2
0.6%

Living arrangements 



 Alone                                                    
40    
12.3%

 In dorms                                                 
13    
4.0%

 With family                                              
272   
83.7%

Monthly income in SDG 



 Less than 500000                                        
277   
85.2%

 500000–1000000                                        
40    
12.3%

 More than 1000000                                      
8     
2.5%





Table 2: Work related demographics
Variable
n
%

Occupation



 Consultant 
4
1.2%

 House officer
90
27.7%

 Lab technician
21
6.5%

 Medical officer 
83
25.5%

 Nurse 
62
19.1%

 Pharmacist
16
4.9%

 Registrar
44
13.5%

 Specialist 
5
1.5%

The field of specialty for Consultants and Specialists and Registrars*



 Internal medicine
8
2.5%

 General surgery
8
2.5%

 Orthopedic surgery
4
1.3%

 Pediatrics
11
3.5%

 Obstetrics and gynecology
12
3.8%

 Radiology or neurology
2
0.6%

Hospital



 Alnau                             
111  
34.2%

 Albuluk                           
86   
26.5%

 Alsuadi                           
70   
21.5%

 Bir Alwaldeen                     
58   
17.8%

Transportation to work 



 Public transportation                                    
186   
57.2%

 Hospital Shuttle services                                
82    
25.2%

 Private vehicle                                          
45    
13.8%

 Walking/Bicycle                                          
12    
3.7%

Crisis management training



 Yes
127   
39.1%

 No
198   
60.9%

*There are 8 missing responses in the field of specialty

Exposure to Terrorist Events and Use of Support Services
With respect to exposure to violence, 60% of participants reported property damage due to terrorism, 21% were physically injured, 49% had a family member injured, and 41% experienced security escalations at their workplace.


Regarding the utilization of support services, 15% used employee daycare services and 35% used hospital-provided sleeping arrangements. Overall, the use of supportive services was limited: only 12% received personal support from hospital staff, 18% participated in workplace group support, and 12% used a community support center. Detailed data are summarized in Table 3.
Table 3: Exposure to Terrorist Events and Use of Support Services
Variable
n
%

Property damaged by terrorism?



 Yes                                                                    
194
59.7%

 No                                                                     
131
40.3%

Physically injured by terrorism?



 Yes                                                                    
67
20.6%

 No                                                                     
258
79.4%

Family members injured by terrorism?



 Yes                                                                    
159
48.9%

 No                                                                     
166
51.1%

Exposed to security escalations at a previous workplace?



 Yes                                                                    
133
40.9%

 No                                                                     
192
59.1%

Used employee daycare services?



 Yes                                                                    
49
15.1%

 No                                                                     
276
84.9%

Used hospital-provided sleeping arrangements?



 Yes                                                                    
113
34.8%

 No                                                                     
212
65.2%

Used personal support from hospital staff?



 Yes                                                                    
39
12.0%

 No                                                                     
286
88.0%

Participated in workplace group support?



 Yes                                                                    
57
17.5%

 No                                                                     
268
82.5%

Used a community support center?



 Yes
39
12%

    No
286
88%

*The questions presented in this table are abbreviated for clarity. The original wording is available upon request 

Attendance at Work During Wartime
Analysis of work attendance showed that a majority of participants (66.77%) maintained their presence at work, while 33.23% reported partial or full absenteeism.
Several demographic and situational factors were significantly associated with work attendance during wartime.
Attendance was significantly higher among males compared to females (74.8% vs. 60.8%; p = 0.011). Single participants had the highest attendance rate (71.0%), while married and widowed participants were more likely to remain at home (p = 0.003).
Participants who were physically injured reported lower attendance compared to those who had not (49.3% vs. 71.3%; p = 0.001). Similarly, having an injured family member was associated with lower attendance (57.9% vs. 75.3%; p = 0.001).
However ,attendance varied significantly by role (p = 0.027), with the highest rates among medical officers (75.9%) and house officers (73.3%), and the lowest among pharmacists (37.5%).
Crisis management training significantly affected the attendance level. Participants who had received training had higher attendance (74.8% vs. 61.6%; p = 0.014). As well, attendance also varied significantly by hospital (p = 0.001), with the highest attendance at Albuluk Hospital (90.7%) and the highest absenteeism at Alsuadi Hospital (51.4%).

Table 4: Attendance at Work During Wartime
Variable
Absenteeism (Full or partial) %(N)
No Absenteeism %(N)

Overall attendance
33.23% (108)
66.77% (217)

Gender



  Male
25.2% (35)
74.8% (104)

  Female
39.2% (73)
60.8% (113)

Marital Status



  Single
29.0% (72)
71.0% (176)

  Married
45.1% (32)
54.9% (39)

  Divorced
33.3% (1)
66.7% (2)

  Widowed
100.0% (3)
0.0% (0)

Occupation



  Consultant
50.0% (2)
50.0% (2)

  House officer
26.7% (24)
73.3% (66)

  Lab technician
47.6% (10)
52.4% (11)

  Medical officer
24.1% (20)
75.9% (63)

  Nurse
40.3% (25)
59.7% (37)

  Pharmacist
62.5% (10)
37.5% (6)

  Registrar
34.1% (15)
65.9% (29)

  Specialist
40.0% (2)
60.0% (3)

Hospital



  Alnau
45.0% (50)
55.0% (61)

  Albuluk
9.3% (8)
90.7% (78)

  Alsuadi
51.4% (36)
48.6% (34)

  Bir Alwaldeen
24.1% (14)
75.9% (44)

Crisis management training



  Yes
25.2% (32)
74.8% (95)

  No
38.4% (76)
61.6% (122)

Physically injured by terrorism



  Yes
50.7% (34)
49.3% (33)

  No
28.7% (74)
71.3% (184)

Family member injured by terrorism



  Yes
42.1% (67)
57.9% (92)

  No
24.7% (41)
75.3% (125)






Resilience
Median resilience scores was 24.00 (IQR = 12.00), with most participants exhibiting low resilience (68.62%) and only 10.46% demonstrating high resilience (Figure 1). Non-parametric analyses identified significant associations between resilience and demographic/work-related factors (Table 5).  
Regarding demographic factors, males were significantly more resilient than females (p < 0.001), and non-parents showed higher resilience than parents (p= 0.024). Resilience increased with higher religiosity (p= 0.006), with the low-religiosity group exhibiting significantly lower resilience than moderate (p = 0.005) and high (p = 0.014) groups.  
Resilience varied significantly by occupation (p < 0.001). Specialists/Registrars were significantly more resilient than Nurses (p= 0.002) and Pharmacist/Lab Technicians (p = 0.014). Medical Officers also outperformed Nurses (p= 0.031).  
Hospital assignment significantly impacted resilience (p < 0.001). Staff at Bir Alwaldeen Hospital reported significantly lower resilience than those at Alsuadi (p= 0.002), Albuluk (p < 0.001), and Alnau (p < 0.001). Staff receiving crisis management training had higher resilience than untrained peers ( p = 0.002), and full work attendance correlated with greater resilience than partial/none (p = 0.001).  
Moreover , non-users of personal support talks (p = 0.004) and group activities (p = 0.004) exhibited significantly higher resilience than users.  



Summary of Non-Parametric Analyses of Factors Associated with Resilience Scores
Variable
n
Median
Test Statistic
Z-Stat
p-value
Effect Size

Gender


U = 9986.50
-3.51
< 0.001
r = 0.20

   Male
139
27.00





   Female
186
22.50





Parenthood


U = 10415.50
-2.26
0.024
r = 0.13

   Non-Parents
206
25.00





   Parents
119
22.00





Religiosity


H(2) = 10.240

0.006
η² = 0.03

   Low
21
17.00





   Moderate
252
25.00





   High
52
26.00





Occupation


H(5) = 24.572

< .001
η² = 0.08

   Specialist/Registrar
49
28.00





   Medical officer
83
26.00





   House officer
90
24.00





   Pharmacist/Lab technician
37
22.00





   Nurse
62
20.00





   Consultant
4
10.00





Hospital


H(3) = 32.486

< 0.001
η² = 0.10

   Alnau
111
28.00





   Albuluk
86
26.00





   Alsuadi
70
23.50





   Bir Alwaldeen
58
19.00





Crisis Management Training


U = 10006.50
-3.11
0.002
r = 0.17

   Received
127
28.00





   Not Received
198
23.00





Work Attendance


U = 14296.50
3.24
0.001
r = 0.18

   Full
217
27.00





   Partial/None
108
21.00





Support Talks


U = 7179.00
2.913
0.004
r = 0.16

   Not Used
286
25.00





   Used
39
19.00





Group Activity


U = 9468.00
2.844
0.004
r = 0.16

   Not Participated
286
25.00





   Participated
57
20.00





Note. Md = Median; n = sample size; U = Mann-Whitney U test statistic; z = z-score; H = Kruskal-Wallis H test statistic; r = effect size correlation coefficient; η² = eta-squared.

Perceived Sense of Danger
The median sense of danger score was 14.00 (IQR = 7.00), as illustrated in Figure 2, which shows that over 70% of participants had a high sense of danger. The associations between sense of danger and various work-related and personal factors are summarized in Table 6.
A Kruskal-Wallis H test indicated a statistically significant difference in sense of danger scores across occupational groups, H(5) = 20.338, p = .001. Post-hoc comparisons with Bonferroni correction revealed that Pharmacists/Lab technicians (Md = 18.00, n = 37) had significantly higher sense of danger scores compared to Consultants (Md = 8.50, n = 4; adj. p = .018), House officers (Md = 14.00, n = 90; adj. p = .005), Nurses (Md = 13.00, n = 62; adj. p = .011), and Medical officers (Md = 13.00, n = 83; adj. p = .038).
Sense of danger also differed significantly by hospital, H(3) = 8.484, p = .037. Staff at Bir Alwaldeen Hospital (Md = 13.00, n = 58) reported a significantly lower sense of danger than staff at Albuluk Hospital (Md = 15.00, n = 86; adj. p = .030).
As shown in Table 6, participants who had experienced security escalation events reported significantly higher sense of danger scores than those who had not. Similarly, those without a child aged 11–17 years (Md = 14.00) reported higher scores than those with a child in that age group (Md = 12.00). Finally, individuals who did not participate in group activities (Md = 14.00) reported higher scores than those who did (Md = 12.00).





Figure 2: Distribution of sense of danger levels among healthcare workers









Table 6: Summary of Non-Parametric Analyses for Sense of Danger Scores
Variable
n
Median
Test Statistic
Z stat
p-value
Effect Size

Occupation


H(5) = 20.338

.001
η² = .06

   Pharmacist/Lab technician
37
18.00





   Specialist/Registrar
49
15.00





   House officer
90
14.00





   Medical officer
83
13.00





   Nurse
62
13.00





   Consultant
4
8.50





Hospital


H(3) = 8.484

.037
η² = .03

   Albuluk
86
15.00





   Alnau
111
14.00





   Alsuadi
70
13.00





   Bir Alwaldeen
58
13.00





Child 11-17 Yrs


U = 5217.50
2.24
.025
r = .12

   Without Child
297
14.00





   With Child
28
12.00





Escalation Exp.


U = 11118.00
-1.99
.047
r = .11

   Exposed
133
14.00





   Not Exposed
192
13.00





Group Activity


U = 9157.50
2.39
.017
r = .13

   Not Participated
268
14.00





   Participated
57
12.00





Note. Md = Median; n = sample size; U = Mann-Whitney U test statistic; z = z-score; H = Kruskal-Wallis H test statistic; r = effect size correlation coefficient; η² = eta-squared.

Relationship Between Resilience and Sense of Danger
A Spearman's rank-order correlation  examining the relationship between resilience and the perceived sense of danger revealed a weak, positive, and statistically significant correlation between the two variables, rho (325) = 0.178, p = 0.001. The coefficient of determination (rho)² was .0317, suggesting that the two variables shared approximately 3.17% of their variance.
Multivariable Analysis:
Resilience:
Linear regression analysis revealed significant predictors of resilience. The model was statistically significant (F = 5.747, p < .001) and accounted for 30.6% of the variance in resilience scores (R² = .306). Higher resilience scores were significantly associated with a stronger perceived sense of danger (p < .001), higher work attendance (p < .001), certain job profiles (p = .001), participation in group activities (p = .004), use of personal support talks (p = .012), and older age groups (p = .048). In contrast, employment at certain hospitals (p < .001) and prior crisis management training (p = .004) were significantly associated with lower resilience scores.

 Sense of Danger:
Linear regression analysis also identified significant predictors of perceived sense of danger. The model was statistically significant (F = 2.145, p = .002) and explained 14.1% of the variance (R² = .141). A higher sense of danger was significantly associated with higher resilience (p < .001), whereas previous exposure to security escalation events was significantly associated with a lower perceived sense of danger (p = .021).

Discussion:
We analyzed resilience, perceived sense of danger, and work attendance during wartime, looking to pinpoint the factors among healthcare professionals (HCPs) in the only operational hospitals within Khartoum state during the fighting.
Healthcare workers’ attendance in conflict areas is influenced by both demographic and situational factors. In our study, the overall attendance was good with two thirds of participants maintaining full attendance. Better attendance was noted among males and single individuals, likely due to lower caregiving responsibilities and personal safety concerns. Similar trends have been documented in Yemen and Sierra Leone [14,15].
Those who did not suffer personal or familial injuries within the context of the conflict were more likely to attend, illustrating the ways violence directly affects workforce participation. Such findings have been documented in Syria, where personal and familial trauma diminished healthcare workers’ capacities to report to work [16,17].
The higher absenteeism at Al-Saudi Hospital may stem from geographic location, administrative issues, or a combination of both. Previous research has linked high-risk area facilities with poor staff attendance and high vacancy rates [18].
In Uganda, studies noted physicians maintained higher presence during crises, so it may be that medical and house officers perceived stronger professional commitment or indispensability when compared with other cadres. They certainly showed better attendance than most [19].
The median resilience score for participants is concerningly low since over two-thirds demonstrated poor resilience. This supports previous research from conflict-perturbed health systems within Gaza, Libya, Israel, and Northern Ethiopia, which documented the enduring impact of insufficient security and relentless working conditions on the psychological functioning and well-being of healthcare personnel [7,20,21,22]
The significant number of participants experiencing direct violence, such as physical injuries and damage to their property, was a key contributing factor. Resilience, as expected, was strongly associated with numerous demographic and occupational factors, including gender, permutation status, spirituality, professional rank, and prior exposure to crisis management training. Among these, males, non-parents, and those identifying as moderately to highly religious displayed greater resilience. 
This aligns with other studies demonstrating that internal belief systems alongside less caregiving responsibilities enhance adaptive capacity to stressors [23,24]
A substantial proportion of participants exhibited a heightened sense of danger, with pharmacists and laboratory technicians reporting the highest scores. This may reflect their comparatively less exposure to occupational stress prior to the escalation of events, in contrast to physicians and nurses, who are more accustomed to high-stress environments. Notably, only a small percentage of participants had prior training in crisis management, which may partially account for the elevated danger perception observed. Furthermore, a heightened sense of danger was positively associated with previous exposure to security escalations, while participants with children aged 11–17 demonstrated comparatively lower danger scores, potentially reflecting a protective desensitization effect or altered risk perception associated with parenthood.
The statistically significant—but weak—positive correlation between sense of danger and resilience was unexpected. While prior studies have suggested that perceived danger generally undermines psychological coping capacity [25], our findings suggest that in high-stress environments, a heightened awareness of risk may coexist with stronger adaptive strategies, particularly for those with prior exposure or formal training. Alternatively, this may indicate the development of hypervigilant coping behaviors, a phenomenon documented in trauma-exposed populations [26].
The findings of this study have important implications for both health system resilience and workforce sustainability in conflict-affected settings. The high levels of perceived danger and the predominance of low resilience among healthcare workers underscore the urgent need for structured psychosocial support and mental health services tailored to frontline staff. The lack of crisis management training among the majority of participants points to a critical gap in preparedness that could be addressed through targeted training initiatives. Additionally, these insights can inform institutional policies related to staff deployment, safety protocols, and supportive supervision in high-risk environments. Strengthening these systems not only safeguards the well-being of healthcare providers but also ensures continuity of care during periods of instability. Future research should explore the long-term psychological impacts of repeated exposure to insecurity and assess the effectiveness of resilience-building interventions.

Limitations: 
There are many limitations to this study. First of all, although all operating governmental hospitals in Khartoum state were included in this study, it was conducted in one state during the war although other states were in conflict too, this may limit the generalizability of results to the whole country. 
Moreover, this study was conducted amid the missile attacks on Omdurman which might have influenced participation rate due to internet connection issues and high workload on HCPs amid the attacks. Additionally, the self-reporting technique used might have influenced participants to present themselves in more socially accepted manners, even though the questionnaire was anonymous.
Finally, the convenient sampling on hospital level may induce selection bias despite stratified random sampling on institutional level. 

Declaration:
Ethics approval and consent to participate
This study was conducted in accordance with the principles of the Declaration of Helsinki. Participation was voluntary, and all participants provided informed consent before data collection. Confidentiality and anonymity were maintained by assigning unique identification codes and securely storing all data. No personally identifiable information was disclosed. Ethical approval was obtained from medical research ethics committee university of al Gadarif, approval number: GU/FM/REC/Q3.7.25.6
Consent for publication
Not applicable.
Availability of data and materials
The datasets used and analyzed during the current study are available from the corresponding author on reasonable request.
Competing interests
The authors declare that they have no competing interests
Funding
This research received no external funding
Authors' contributions
MM and AM conceived the study and developed the overall concept.
HA, LO, SH, MM, and AM jointly designed the study and developed the methodology.
SH carried out the data analysis.
AA, EM, EA, MO, and HH were responsible for data collection.
All authors took part in drafting the manuscript, critically reviewed and approved the final version, and agreed to be accountable for the integrity of the work.
Acknowledgement
It was conducted under the supervision of Student association of medical education and research (SAMER) university of Khartoum faculty of medicine

References:
Murthy RS, Lakshminarayana R. Mental health consequences of war: a brief review of research findings. World Psychiatry. 2006 Feb;5(1):25–30. 
Scholte WF, Olff M, Ventevogel P, de Vries GJ, Jansveld E, Cardozo BL, et al. Mental health symptoms following war and repression in eastern Afghanistan. JAMA. 2004 Aug 4;292(5):585–93. 
Yates TM, Tyrell FA, Masten AS. Resilience Theory and the Practice of Positive Psychology From Individuals to Societies. 2015. 
Witter S, Wurie H, Chandiwana P, Namakula J, So S, Alonso-Garbayo A, et al. How do health workers experience and cope with shocks? Learning from four fragile and conflict-affected health systems in Uganda, Sierra Leone, Zimbabwe and Cambodia. Health Policy Plan. 2017;32:iii3–13. 
Kimhi S, Marciano H, Eshel Y. The Israel Resilience Index. 2018. 
Afzal MH, Jafar AJN. A scoping review of the wider and long-term impacts of attacks on healthcare in conflict zones. Med Confl Surviv. 2019 Mar;35(1):43–64. 
Sberro-Cohen S, Amit I, Barenboim E, Roitman A. Resilience, sense of danger, and reporting in wartime: a cross-sectional study of healthcare personnel in a general hospital. Hum Resour Health. 2023 Dec 1;21(1). 
Ali. The War in Sudan: How Weapons and Networks Shattered a Power Struggle Zur Verfügung gestellt in Kooperation mit / provided in cooperation with: GIGA German Institute of Global and Area Studies. Available from: www.ssoar.info
Dafallah A, Elmahi OKO, Ibrahim ME, Elsheikh RE, Blanchet K. Destruction, disruption and disaster: Sudan’s health system amidst armed conflict. Vol. 17, Conflict and Health. BioMed Central Ltd; 2023. 
Elnakib S, Elaraby S, Othman F, BaSaleem H, Abdulghani AlShawafi NA, Saleh Al-Gawfi IA, et al. Providing care under extreme adversity: The impact of the Yemen conflict on the personal and professional lives of health workers. Soc Sci Med. 2021 Mar 1;272. 
Zhang Y, LePine JA, Buckman BR, Wei F. Well-being-oriented human resource management practices and employee performance in the Chinese banking sector: The role of social climate and resilience. Hum Resour Manage. 2019;58(1):85–97. doi:10.1002/hrm.21934.
Connor KM, Davidson JRT. Development of a new resilience scale: The Connor-Davidson Resilience Scale (CD-RISC). Depression Anxiety. 2003;18(2):76–82. https://doi.org/10.1002/da.10113.
Solomon Z, Prager E. Eldery Israeli holocaust survivors during the Persian Gulf War: a study of psychological distress. Am J Psychiatry. 1992;149(12):1707–10.
Khader YS, Samhouri D, Aldalaykeh M, Al-Farsi S, Al Nsour M, Abubakar A. Factors affecting healthcare workers’ willingness to work during pandemics: a cross-sectional study. BMC Public Health. 2015;15:498.
Wurie HR, Samai M, Witter S. Retention of health workers in rural Sierra Leone: findings from life histories. Hum Resour Health. 2016;14(1):3.
Fouad FM, Sparrow A, Tarakji A, Alameddine M, El-Jardali F, Coutts A, et al. Health workers and the weaponisation of health care in Syria: a preliminary inquiry for The Lancet–American University of Beirut Commission on Syria. Lancet. 2017;390(10111):2516–26.
Witter S, Wurie H, Chandiwana P, Namakula J, So S, Vong S, et al. Providing health care under extreme adversity: lessons from health workers in conflict-affected areas. Reprod Health Matters. 2017;25(51):114–24.
Pavignani E, Colombo S. Analysing disrupted health sectors: A modular manual. Geneva: World Health Organization; 2009.
Ssengooba F, McPake B, Namakula J, Hongoro C, Rutebemberwa E, Maseko F, et al. Health systems reforms in Uganda: adoption and implementation of policies for retention of health workers in remote and rural areas. BMC Health Serv Res. 2015;15:347.
Alnassar M, Al-Jawaldeh H, Ayoub H. Mental health impact of conflict on frontline workers in the Gaza Strip: A mixed-methods study. Conflict and Health. 2023;17(1):47.
Abugraga A, Elmabrouk A, Ismail A. Health workforce stress and resilience during armed conflict in Libya: a qualitative study. BMJ Open. 2023;13(1):e066731.
Bashir M, et al. Religiosity and mental health resilience among healthcare workers in war-torn regions of Ethiopia. Int J Ment Health Syst. 2022;16(1):59.
Zaman S, Raza A, Fatima H. Parental status, religiosity, and resilience among Pakistani emergency workers during crises. BMC Psychiatry. 2023;23:112.
Saleh Y, Al-Khatib A. Barriers to mental health service utilization among HCWs in conflict settings: A scoping review. J Glob Health Rep. 2024;8:e2024001.
Tadesse Y, et al. Exposure to conflict, perception of danger, and psychological coping in healthcare providers in Northern Ethiopia. Soc Psychiatry Psychiatr Epidemiol. 2023.
Stewart TJ, Kim H, Lopez C. Hypervigilance and adaptive coping in trauma-exposed medical staff: Evidence from post-disaster zones. Int J Stress Manag. 2022;29(4):257–270.
