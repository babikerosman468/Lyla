\documentclass[12pt]{article}
\usepackage[utf8]{inputenc}
\usepackage{geometry}
\usepackage{hyperref}
\usepackage{booktabs}
\usepackage{listings}
\usepackage{xcolor}
\geometry{margin=1in}

\lstdefinelanguage{SAS}{
  morekeywords={data,set,run,proc,means,univariate,print,format,call,streaminit,rand,do,if,then,else,end,output,array,retain,drop,keep,where,by,merge,sort,title,footnote,quit,macro,mend,%let,%do,%end,%if,%then,%else,%sysfunc},
  sensitive=false,
  morecomment=[l]*,
  morestring=[b]"
}
\lstset{
  language=SAS,
  basicstyle=\ttfamily\small,
  keywordstyle=\color{blue!70!black}\bfseries,
  commentstyle=\color{gray!70},
  stringstyle=\color{green!50!black},
  showstringspaces=false,
  frame=single,
  breaklines=true,
  columns=fullflexible
}

\title{SAS Simulation for the Connor--Davidson Resilience Scale (CD\,-\,RISC)}
\author{}
\date{}

\begin{document}
\maketitle

\section*{Purpose}
This script simulates CD-RISC item responses on the 0--4 Likert scale and computes total scores for the 25-, 10-, and 2-item versions. It supports:
\begin{itemize}
  \item Pre/post designs with a user-defined effect size on resilience.
  \item Optional correlated external variables (e.g., stress) for validity checks.
  \item Reliability summaries (Cronbach's $\alpha$) and distribution diagnostics.
\end{itemize}

\section*{How the simulation works (brief)}
Each subject has a latent resilience trait $\theta \sim \mathcal{N}(\mu,\sigma^2)$. Item $j$ draws a continuous response
\[
y_{ij} = a_j\,\theta_i + \varepsilon_{ij},\quad \varepsilon_{ij}\sim \mathcal{N}(0,\tau_j^2),
\]
which is discretized into 5 ordered categories (0--4) via fixed cutpoints. Discriminations $a_j$ vary modestly across items. Post--intervention shifts apply $\theta_{\text{post}} = \theta_{\text{pre}} + \Delta$, where $\Delta$ sets the standardized effect size.

\section*{Plug-and-Play SAS Code}

\subsection*{1) Master macro: simulate data and score CD-RISC}
\begin{lstlisting}
%macro simulate_cdrisc(
    n              = 500,        /* sample size per arm/time point        */
    seed           = 20250823,   /* RNG seed                              */
    mu             = 0,          /* mean latent resilience at baseline    */
    sigma          = 1,          /* SD of latent resilience               */
    effect_size    = 0.35,       /* standardized pre->post shift (Cohen d)*/
    design         = prepost,    /* none | prepost | twoarm_prepost       */
    miss_rate      = 0.03,       /* MCAR missingness at item level        */
    add_validity   = YES,        /* add negative stress, positive wellbeing*/
    out            = work.cdrisc /* output table                          */
  );
/*-------------------------- Item blueprints -------------------------------*/
/* Item counts for versions */
%let k25 = 25;
%let k10 = 10;
%let k02 = 2;

/* Index sets for CD-RISC-10 and CD-RISC-2 (using common published subsets;
   item text is not reproduced; only indices are used for scoring). 
   Adjust indices if your institution uses different subsets.               */
%let idx10 = 1 4 6 7 8 11 14 16 17 19;   /* 10 distinct items from the 25 */
%let idx02 = 1 8;                        /* 2-item form (adapt/bounce back)*/

/* Fixed cutpoints (0..4) applied to continuous y: 
   category 0: (-inf, -1.5], 1: (-1.5,-0.5], 2: (-0.5,0.5], 3: (0.5,1.5], 4: (1.5, inf) */
data _cutpoints;
  c1 = -1.5; c2 = -0.5; c3 = 0.5; c4 = 1.5;
run;

/* Item discriminations a_j and residual scales tau_j.
   We vary a_j modestly to create realistic item-total spread. */
data _items;
  call streaminit(&seed);
  do j = 1 to &k25;
    a  = rand("uniform")*0.6 + 0.6; /* a_j in [0.6,1.2] */
    t  = sqrt( max(0.25, 1.0 - a*a) ); /* residual SD; floor to avoid too-low noise */
    output;
  end;
run;

/* Helper: create long design frame depending on requested design */
data _design;
  length arm $8 time $6;
  if upcase("&design") = "NONE" then do;
    arm = "single"; time="single";
    do id = 1 to &n; output; end;
  end;
  else if upcase("&design") = "PREPOST" then do;
    arm = "single";
    do id = 1 to &n;
      time="pre";  output;
      time="post"; output;
    end;
  end;
  else if upcase("&design") = "TWOARM_PREPOST" then do;
    do arm = "control","treat";
      do id = 1 to &n;
        time="pre";  output;
        time="post"; output;
      end;
    end;
  end;
  else do;
    put "ERROR: Unknown design=&design";
    stop;
  end;
run;

/* Simulate latent theta by arm/time, applying effect_size where appropriate */
proc sql noprint;
  select c1, c2, c3, c4 into :c1,:c2,:c3,:c4 from _cutpoints;
quit;

data _sim_long;
  merge _design;
  by id;
  call streaminit(&seed);
  length version $8;
  /* baseline latent */
  theta_base = rand("normal", &mu, &sigma);
  /* apply design-specific shifts */
  delta = 0;
  select (upcase("&design"));
    when ("NONE")           delta = 0;
    when ("PREPOST")        delta = (time="post")*&effect_size;
    when ("TWOARM_PREPOST") delta = (time="post" and arm="treat")*&effect_size;
    otherwise delta = 0;
  end;
  theta = theta_base + delta;

  /* simulate 25 items -> discretize -> missingness */
  array itemc[&k25] 8.; /* continuous */
  array item [&k25] 8.; /* 0..4 categorical */
  /* load item params */
  if _n_ = 1 then do;
    dcl hash H(dataset:"_items"); H.defineKey("j"); H.defineData("a","t"); H.defineDone();
  end;
  do j=1 to &k25;
    rc=H.find();
    /* continuous response */
    itemc[j] = a*theta + rand("normal", 0, t);
    /* discretize */
    if      itemc[j] <=  &c1 then item[j]=0;
    else if itemc[j] <=  &c2 then item[j]=1;
    else if itemc[j] <=  &c3 then item[j]=2;
    else if itemc[j] <=  &c4 then item[j]=3;
    else                          item[j]=4;
    /* MCAR missingness */
    if rand("uniform") < &miss_rate then item[j]=.;
  end;

  /* --- Totals by version --- */
  /* CD-RISC-25 */
  total25 = 0; nobs25=0;
  do j=1 to &k25; if item[j] ne . then do; total25+item[j]; nobs25+1; end; end;
  range25 = 4*&k25; /* 0..100 */
  score25 = total25; /* 0..100 */

  /* CD-RISC-10 subset */
  total10 = 0; nobs10=0;
  %local i val;
  %let i=1;
  %do %while(%scan(&idx10,&i) ne );
    %let val=%scan(&idx10,&i);
    if item[&val] ne . then do; total10+item[&val]; nobs10+1; end;
    %let i=%eval(&i+1);
  %end;
  range10 = 4*&k10; /* 0..40 */
  score10 = total10;

  /* CD-RISC-2 subset */
  total02 = 0; nobs02=0;
  %let i=1;
  %do %while(%scan(&idx02,&i) ne );
    %let val=%scan(&idx02,&i);
    if item[&val] ne . then do; total02+item[&val]; nobs02+1; end;
    %let i=%eval(&i+1);
  %end;
  range02 = 4*&k02; /* 0..8 */
  score02 = total02;

  /* Optional validity variables: stress (neg), wellbeing (pos) */
  %if %upcase(&add_validity)=YES %then %do;
    stress     = rand("normal", 0, 1) - 0.60*theta;     /* higher theta -> lower stress */
    wellbeing  = rand("normal", 0, 1) + 0.55*theta;     /* higher theta -> higher WB    */
  %end;

  /* Keep wide item responses too */
  %do j=1 %to &k25; item&j = item[&j]; %end;

  /* labels */
  label
    theta    = "Latent resilience"
    score25  = "CD-RISC-25 total (0-100)"
    score10  = "CD-RISC-10 total (0-40)"
    score02  = "CD-RISC-2 total (0-8)"
    nobs25   = "# observed items (25)"
    nobs10   = "# observed items (10)"
    nobs02   = "# observed items (2)"
    stress   = "External stress (lower is better)"
    wellbeing= "External wellbeing (higher is better)"
  ;
  output;
  drop j rc a t itemc:;
run;

/* Output */
data &out; set _sim_long; run;
%mend simulate_cdrisc;
\end{lstlisting}

\subsection*{2) Example: simple cross-sectional simulation}
\begin{lstlisting}
/* One-wave sample, moderate reliability, small missingness */
%simulate_cdrisc(n=600, design=none, miss_rate=0.02, effect_size=0, out=work.cdrisc1);

/* Distribution summaries */
proc means data=work.cdrisc1 n mean std min p25 median p75 max;
  var score25 score10 score02;
run;

/* Reliability (Cronbach's alpha) for CD-RISC-25 */
proc corr data=work.cdrisc1 alpha nomiss;
  var item1-item25;
run;

/* Convergent validity checks */
proc corr data=work.cdrisc1;
  var score25 score10 score02 stress wellbeing;
run;
\end{lstlisting}

\subsection*{3) Example: pre/post single-arm with improvement}
\begin{lstlisting}
%simulate_cdrisc(n=300, design=prepost, effect_size=0.45, out=work.cdrisc2);

/* Wide-to-long for plotting means by time */
proc means data=work.cdrisc2 n mean std;
  class time;
  var score25 score10 score02;
run;

/* Paired t-test for pre vs post (25-item) */
proc sort data=work.cdrisc2; by id time; run;
data prepost;
  merge work.cdrisc2(where=(time="pre")  rename=(score25=pre25))
        work.cdrisc2(where=(time="post") rename=(score25=post25));
  by id;
run;

proc ttest data=prepost;
  paired pre25*post25;
run;
\end{lstlisting}

\subsection*{4) Example: two-arm pre/post randomized design}
\begin{lstlisting}
%simulate_cdrisc(n=250, design=twoarm_prepost, effect_size=0.35, out=work.cdrisc3);

/* Check group x time means */
proc means data=work.cdrisc3 mean std;
  class arm time;
  var score25;
run;

/* Difference-in-differences using PROC GLM */
proc glm data=work.cdrisc3;
  class arm time;
  model score25 = arm|time;
  lsmeans arm*time / pdiff cl;
run; quit;
\end{lstlisting}

\subsection*{5) Notes and tuning tips}
\begin{itemize}
  \item \textbf{Item difficulty}: Adjust cutpoints \texttt{c1--c4} to shift endorsement rates.
  \item \textbf{Reliability}: Increase discriminations \(a_j\) or decrease residual SD \( \tau_j\) to raise internal consistency.
  \item \textbf{Missingness}: \texttt{miss\_rate} currently MCAR; extend to MAR by making missingness a function of \(\theta\) if desired.
  \item \textbf{Subsets}: If your CD-RISC-10/2 use different item indices, update \texttt{\%let idx10} and \texttt{\%let idx02}.
  \item \textbf{Scaling}: Raw totals already match common score ranges (0--100, 0--40, 0--8). Apply rescaling only if you change item counts.
\end{itemize}

\section*{Output tables you will get}
\begin{itemize}
  \item Descriptive stats for each version's total score.
  \item Cronbach's $\alpha$ for the 25-item pool.
  \item Correlations with external validity variables (\texttt{stress}, \texttt{wellbeing}).
  \item For pre/post: paired tests and/or group$\times$time contrasts.
\end{itemize}

\section*{Attribution}
This code simulates anonymized item responses consistent with the CD-RISC structure without reproducing proprietary item content. For the official scales and permissions, visit the CD-RISC website.

\end{document}


