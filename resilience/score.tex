\documentclass[12pt]{article}
\usepackage{amssymb}
\usepackage{geometry}
\usepackage{graphicx}
\usepackage{wrapfig}
\usepackage{array}
\usepackage{booktabs}
\usepackage{xcolor}
\usepackage{hyperref}
\usepackage{amsmath}
\usepackage{lipsum}
\usepackage{caption}
\usepackage{ulem}
\usepackage{xcolor}
\usepackage{tikz}
\usetikzlibrary{arrows.meta, positioning, shapes}

\usepackage{sectsty}
\sectionfont{\color{sectionColor}}       % Color for \section
\subsectionfont{\color{subsectionColor}}


\definecolor{titleColor}{HTML}{800000}        % Maroon
\definecolor{sectionColor}{HTML}{4682B4}      % SteelBlue
\definecolor{textHighlight}{HTML}{DAA520}     % Goldenrod
\definecolor{backgroundColor}{HTML}{F0F8FF}   % AliceBlue
\definecolor{emphasisColor}{HTML}{228B22}     % ForestGreen
\definecolor{subsectionColor}{HTML}{6A5ACD}   % SlateBlue


\title{\textcolor{titleColor}{\textbf{\Huge \textbf{ Connor-Davidson Resilience Scale (CD-RISC) }}}}

\author{Laila Osman\thanks{\texttt{Lylababiker@gmail.com}, University of Khartoum, Khartoum, Sudan}} 
\date{\today}
\begin{document}
\maketitle

\section*{ Overview }

The \textbf{Connor-Davidson Resilience Scale (CD-RISC)} is a psychometric tool developed by Kathryn M. Connor and Jonathan R.T. Davidson to assess resilience, defined as the ability to ``thrive in the face of adversity.'' It was created to address the lack of generalizable resilience measures and has since become widely used in clinical, research, and community settings. The scale measures resilience as a multidimensional construct, encompassing traits like adaptability, perseverance, stress tolerance, and recovery from hardship.

\section*{🔢 Versions of the CD-RISC}

There are three authorized versions of the CD-RISC, each designed for specific contexts:

\begin{enumerate}
    \item \textbf{CD-RISC-25 (25 items)}: The original scale, yielding a total score between 0 and 100. It assesses five factors:
    \begin{itemize}
        \item Personal competence, tenacity, and high standards.
        \item Trust in instincts and tolerance of negative affect.
        \item Positive acceptance of change and secure relationships.
        \item Control.
        \item Spiritual influences.
    \end{itemize}
    
    \item \textbf{CD-RISC-10 (10 items)}: A shortened, unidimensional version developed to improve factor stability and practicality. Focuses on core resilience traits like adaptability, perseverance, and stress management.
    
    \item \textbf{CD-RISC-2 (2 items)}: An ultra-brief version often used for rapid assessment or tracking changes over time. Includes items related to adaptability and bouncing back from hardship.
\end{enumerate}

\subsection*{📊 Key Features of CD-RISC Versions}

\begin{center}
\begin{tabular}{>{\bfseries}l c c c l}
\toprule
Version & Items & Score Range & Primary Use & Key Traits Measured \\
\midrule
CD-RISC-25 & 25 & 0--100 & Comprehensive assessment & Competence, trust, acceptance, control, spirituality \\
CD-RISC-10 & 10 & 0--40 & General research / screening & Adaptability, perseverance, stress tolerance, recovery \\
CD-RISC-2 & 2 & 0--8 & Brief assessment / monitoring & Adaptability and bouncing back from adversity \\
\bottomrule
\end{tabular}
\end{center}

\section*{📝 Scale Structure and Scoring}

\begin{itemize}
    \item \textbf{Response Format}: Items are rated on a \textbf{5-point Likert scale} ranging from 0 (``Not true at all'') to 4 (``True nearly all the time'').
    \item \textbf{Scoring}: Total scores are summed, with higher scores indicating greater resilience.
    \item \textbf{Factors}: The CD-RISC-25 includes five factors, while the CD-RISC-10 and CD-RISC-2 are unidimensional, focusing on overall resilience.
\end{itemize}

\section*{✔️ Psychometric Properties}

The CD-RISC has demonstrated strong reliability and validity across diverse populations:

\begin{itemize}
    \item \textbf{Reliability}:
    \begin{itemize}
        \item High internal consistency (Cronbach’s $\alpha$: 0.88--0.90 for CD-RISC-10).
        \item Test-retest reliability indicates stability over time.
    \end{itemize}
    \item \textbf{Validity}:
    \begin{itemize}
        \item \textbf{Convergent Validity}: Correlates positively with measures of hardiness and negatively with perceived stress and vulnerability.
        \item \textbf{Predictive Validity}: Predicts treatment outcomes and mental health diagnoses (e.g., lower resilience scores are associated with higher attrition in military settings and greater severity of psychiatric symptoms).
        \item \textbf{Construct Validity}: Supported by factor analyses and associations with related constructs like depression and life satisfaction.
    \end{itemize}
\end{itemize}

\section*{🌍 Generalizability and Applications}

The CD-RISC has been validated across various cultures (e.g., Korean, Chinese, German, Spanish) and populations, including:
\begin{itemize}
    \item Clinical groups (e.g., PTSD, anxiety, depression, substance use disorders).
    \item Community and general populations.
    \item Specific contexts like chronic illness, trauma, and military settings.
\end{itemize}

It is used to:
\begin{itemize}
    \item Assess resilience in research and clinical practice.
    \item Evaluate intervention effectiveness (e.g., therapy outcomes).
    \item Identify individuals at risk for stress-related disorders.
\end{itemize}

\section*{📋 How to Access the Scale}

The CD-RISC is \textbf{copyright-protected}. To obtain the official scale, instructions, and permissions for use, visit the \href{https://www.connordavidson-resiliencescale.com/}{official CD-RISC website}.

\section*{💡 Comparison with Other Scales}

The \textbf{Brief Resilience Scale (BRS)} is another common tool but differs from the CD-RISC:
\begin{itemize}
    \item The BRS focuses narrowly on the ability to ``bounce back'' from stress.
    \item The CD-RISC assesses broader resources and traits (e.g., adaptability, spirituality, social support).
    \item Studies show the CD-RISC and BRS are correlated but capture distinct aspects of resilience.
\end{itemize}

\section*{💎 Conclusion}

The CD-RISC is a robust, versatile tool for measuring resilience. Its strong psychometric properties and adaptability across cultures and contexts make it valuable for research and clinical use. Users should select the version (25-, 10-, or 2-item) based on their needs and ensure proper authorization from the official website.

For more details, refer to the \href{https://www.connordavidson-resiliencescale.com/}{official CD-RISC website} or published studies.

\end{document}

Laila Osman\thanks{\texttt{Lylababiker@gmail.com}, University of Khartoum, Khartoum, Sudan} \\

