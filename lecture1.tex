\documentclass[12pt]{article}
\usepackage{amssymb}
\usepackage{geometry}
\usepackage{graphicx}
\usepackage{wrapfig}
\usepackage{amsmath}
\usepackage{lipsum}
\usepackage{caption}
\usepackage{ulem}
\usepackage{xcolor}
\usepackage{tikz}
\usetikzlibrary{arrows.meta, positioning, shapes}

\usepackage{sectsty}
\sectionfont{\color{sectionColor}}       % Color for \section
\subsectionfont{\color{subsectionColor}}


\definecolor{titleColor}{HTML}{800000}        % Maroon
\definecolor{sectionColor}{HTML}{4682B4}      % SteelBlue
\definecolor{textHighlight}{HTML}{DAA520}     % Goldenrod
\definecolor{backgroundColor}{HTML}{F0F8FF}   % AliceBlue
\definecolor{emphasisColor}{HTML}{228B22}     % ForestGreen
\definecolor{subsectionColor}{HTML}{6A5ACD}   % SlateBlue


\title{\textcolor{titleColor}{\textbf{\Huge \textbf{ 

 }}}}

\author{B. M. Osman\thanks{Al Nahada University, Cairo \\ \texttt{babikerosman@yahoo.com}}}
\date{\today}
\begin{document}
\maketitle

Chi-square Tests: Theory, SAS, and Genetics Applications

\section*{Introduction}
The Chi-square test (\chi^2) is a statistical method to compare observed frequencies with expected frequencies under a certain hypothesis.

\section*{Chi-square Formula}
\[
\chi^2 = \sum \frac{(O - E)^2}{E}
\]
where O = observed count, E = expected count.

\section*{Uses}
- Test goodness-of-fit
- Test independence between categorical variables
- Test genetic hypotheses (e.g., Mendelian ratios, Hardy-Weinberg equilibrium)

\section*{Examples}

\subsection*{1. Fair Die Test}
Test if a die is fair (all faces equally likely).

\subsection*{2. Coin Toss Test}
Test if a coin is fair (heads vs tails 50-50).

\subsection*{3. Mendelian Genetics Test}
Test if pea color fits expected 3:1 ratio.

\subsection*{4. Hardy-Weinberg Equilibrium Test}
Test if genotype frequencies fit HW expected ratios.

\section*{SAS Code Examples}
See separate SAS scripts provided.

\end{document}


\documentclass{beamer}
\title{Chi-square Tests in Research}

\frame{\titlepage}

\begin{frame}{What is Chi-square?}
\[
\chi^2 = \sum \frac{(O - E)^2}{E}
\]
\end{frame}

\begin{frame}{Uses of Chi-square Tests}
\begin{itemize}
\item Goodness-of-fit
\item Test of independence
\item Genetic hypothesis testing
\item Hardy-Weinberg equilibrium testing
\end{itemize}
\end{frame}

\begin{frame}{Example: Fair Die Test}
Data: counts for faces 1 to 6 \\
Expected: equal probabilities (16.67\%)
\end{frame}

\end{document}


3. dice.sas

data dice;
  input Face Count;
  datalines;
1 8
2 12
3 9
4 11
5 10
6 10
;
run;

proc freq data=dice;
  tables Face / chisq testp=(16.67 16.67 16.67 16.67 16.67 16.67);
  weight Count;
run;


---

4. coin.sas

data coin;
  input Side $ Count;
  datalines;
Heads 58
Tails 42
;
run;

proc freq data=coin;
  tables Side / chisq testp=(50 50);
  weight Count;
run;


---

5. mendel.sas

data peas;
  input Color $ Count;
  datalines;
Yellow 290
Green 110
;
run;

proc freq data=peas;
  tables Color / chisq testp=(75 25);
  weight Count;
run;


---

6. hardy_weinberg.sas

data hw;
  input Genotype $ Count;
  datalines;
AA 50
Aa 30
aa 20
;
run;

proc freq data=hw;
  tables Genotype / chisq testp=(49 42 9);
  weight Count;
run;


---

7. dice.csv

Face,Count
1,8
2,12
3,9
4,11
5,10
6,10


---

8. coin.csv

Side,Count
Heads,58
Tails,42


---

9. mendel.csv

Color,Count
Yellow,290
Green,110


---

10. hardy_weinberg.csv

Genotype,Count
AA,50
Aa,30
aa,20


---

11. quiz.tex


\section*{Chi-square Quiz}

1. Write the Chi-square formula. \\
2. When do you use the goodness-of-fit test? \\
3. What are degrees of freedom for 4 categories? \\

\end{document}


12. answer_key.tex

\documentclass[12pt]{article}
\begin{document}

\section*{Answer Key}

1. \chi^2 = \sum \frac{(O - E)^2}{E} \\
2. To compare observed vs. expected frequencies in one categorical variable. \\
3. Degrees of freedom = 3 (number of categories - 1) \\

\end{document}


---

13. README.md

# Chi-square Teaching Module

## Contents

- chi_square_handout.tex : Full handout LaTeX source
- chi_square_slides.tex : Beamer slides LaTeX source
- dice.sas : SAS script for fair die test
- coin.sas : SAS script for fair coin test
- mendel.sas : SAS script for Mendelian ratio test
- hardy_weinberg.sas : SAS script for HW equilibrium test
- dice.csv : Example CSV data for dice
- coin.csv : Example CSV data for coin
- mendel.csv : Example CSV data for Mendelian peas
- hardy_weinberg.csv : Example CSV data for HW
- quiz.tex : Quiz LaTeX source
- answer_key.tex : Answer key LaTeX source

## Instructions

1. Compile .tex files using pdflatex or your LaTeX editor.
2. Run .sas files in SAS or SAS University Edition.
3. Use CSVs to import data if needed.
4. Edit author/title as needed.


---

If you want me to generate scripts or instructions for zipping these, just say so!
Otherwise, copy these into files with the exact names, and you have your full teaching module ready.

Need help with anything else?


