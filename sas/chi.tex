\documentclass[12pt]{article}
\usepackage{amssymb}
\usepackage{geometry}
\usepackage{graphicx}
\usepackage{wrapfig}
\usepackage{amsmath}
\usepackage{lipsum}
\usepackage{caption}
\usepackage{ulem}
\usepackage{xcolor}
\usepackage{tikz}
\usetikzlibrary{arrows.meta, positioning, shapes}

\usepackage{sectsty}
\sectionfont{\color{sectionColor}}       % Color for \section
\subsectionfont{\color{subsectionColor}}


\definecolor{titleColor}{HTML}{800000}        % Maroon
\definecolor{sectionColor}{HTML}{4682B4}      % SteelBlue
\definecolor{textHighlight}{HTML}{DAA520}     % Goldenrod
\definecolor{backgroundColor}{HTML}{F0F8FF}   % AliceBlue
\definecolor{emphasisColor}{HTML}{228B22}     % ForestGreen
\definecolor{subsectionColor}{HTML}{6A5ACD}   % SlateBlue


\title{\textcolor{titleColor}{\textbf{\Huge \textbf{Understanding the Chi-square Test}}}}

\author{B. M. Osman\thanks{Al Nahada University, Cairo \\ \texttt{babikerosman@yahoo.com}}}
\date{\today}
\begin{document}
\maketitle


\section*{What is Chi-square?}

The Chi-square ($\chi^2$) test is a statistical method used to compare observed data with data we would expect to obtain according to a specific hypothesis. It measures the difference between expected and observed frequencies.

\section*{When to use it}

\begin{itemize}
  \item \textbf{Goodness-of-fit:} Does the sample data fit a specified distribution?
  \item \textbf{Test of independence:} Are two categorical variables independent?
  \item \textbf{Test of homogeneity:} Are distributions the same across populations?
\end{itemize}

\section*{SAS Code Example}

Below is an example of a Chi-square test of independence using SAS:

\begin{verbatim}
data example;
   input Gender $ Preference $ Count;
   datalines;
Male   A 30
Male   B 20
Female A 25
Female B 25
;
run;

proc freq data=example;
   tables Gender*Preference / chisq;
   weight Count;
run;
\end{verbatim}

This tests whether \texttt{Gender} and \texttt{Preference} are independent.

\end{document}


---

If you’d like, I can help adapt this to goodness-of-fit, real data, or help you run it in SAS Studio. Just let me know!


