\documentclass[12pt]{article}
\usepackage[utf8]{inputenc}
\usepackage{amssymb}
\usepackage{geometry}
\usepackage{graphicx}
\usepackage{wrapfig}
\usepackage{amsmath}
\usepackage{lipsum}
\usepackage{caption}
\usepackage{ulem}
\usepackage{xcolor}
\usepackage{polyglossia}
\usepackage{tikz}
\usetikzlibrary{arrows.meta, positioning, shapes}
\setdefaultlanguage{arabic}
\setotherlanguage{english}

% Font and layout settings
\newfontfamily\arabicfont[Script=Arabic,Scale=1.2]{Amiri}
\geometry{top=2cm, bottom=2cm, left=2cm, right=2cm}

% Colors
\definecolor{titleColor}{HTML}{800000} % Maroon for titles
\definecolor{sectionColor}{HTML}{4682B4} % SteelBlue for sections
\definecolor{textHighlight}{HTML}{DAA520} % Goldenrod for highlights
\definecolor{backgroundColor}{HTML}{F0F8FF} % AliceBlue background
\definecolor{emphasisColor}{HTML}{228B22} % ForestGreen for emphasis

\title{تحليل شخصية بينوكيو}
\author{\textbf{ليلي بابكر}\\    \small باحثة  ﻒﻳ ﻦﻇﺮﻳﺓ الوعى والبيولوجيا العصبية  \thanks{\textenglish{Khartoum University, Cairo \\ \texttt{lylababiker@yahoo.com}}}       }                                
 \date{\today}
\begin{document}

\maketitle

\section{مقدمة}

\begin{flushright} "بينوكيو" هي شخصية خيالية شهيرة ظهرت لأول مرة في رواية "مغامرات بينوكيو" التي كتبها الكاتب الإيطالي كارلو كولّودي عام 1883. تحوّلت هذه الشخصية إلى رمز عالمي للأطفال والصدق والنمو الأخلاقي. \end{flushright}

\section{الوصف العام للشخصية}

\begin{flushright} بينوكيو هو دمية خشبية صنعها نجار يُدعى جيبيتو، وقد منحته جنية زرقاء الحياة وأملته بأن يتحوّل إلى ولد حقيقي إذا أثبت صدقه وشجاعته وحسن سلوكه. \end{flushright}

\section{السمات البارزة}

\begin{itemize} \item \textbf{الأنف الذي يطول عندما يكذب:} يرمز إلى أهمية الصدق. \item \textbf{الفضول والاندفاع:} يقوده إلى الكثير من التحديات. \item \textbf{التطور الأخلاقي:} رغم أخطائه يتعلّم من تجاربه. \item \textbf{رمز للطفولة:} تصرفاته تمثل براءة الأطفال وسعيهم للنضج. \end{itemize}

\section{الدلالات الرمزية}

\begin{flushright} رحلة بينوكيو تعكس رحلة الإنسان نحو النضج، وتحوله من كائن مادي إلى إنسان مكتمل. الجنية الزرقاء ترمز إلى الضمير أو القوة التي تدفعه نحو الخير. \end{flushright}

\section{التحليل النفسي}

\begin{flushright} يمكن فهم شخصية بينوكيو من منظور التحليل النفسي بوصفها تمثل صراع الأنا بين الرغبات الطفولية والنضج الأخلاقي. الكذب ونتائجه الفورية (طول الأنف) هو تمثيل مباشر للشعور بالذنب. شخصية جيبيتو تمثل الأب الحنون، بينما تمثل الجنية الزرقاء دور الأم المثالية أو الأنا العليا. \end{flushright}

\section{التطبيقات التربوية}

\begin{flushright} قصة بينوكيو تُستخدم في التعليم لتعليم الأطفال الصدق والمسؤولية والتفكير في عواقب الأفعال. الرموز البسيطة مثل طول الأنف تترك أثراً تربوياً مباشراً، ما يجعلها أداة فعالة في التربية الأخلاقية. \end{flushright}

\section{خاتمة}

\begin{flushright} بينوكيو ليست مجرد قصة للأطفال، بل هي مرآة للرحلة الإنسانية نحو الحقيقة والنمو. من خلال رموزها العميقة، تعلّم القارئ أهمية الصدق، والتعلم من الخطأ، والقدرة على التحول الداخلي. \end{flushright}

\end{document}



